\documentclass{article}
\usepackage[utf8]{inputenc}

\title{Aritha - Rapport}
\author{Simon Lukowski}
\date{Octobre 2022}

\begin{document}

\maketitle

\section{Lexer et Parser}



L'utilisation de Ocamllex et yacc a rendu cette étape très facile, la seule difficulté était de lire la documentation yacc pour comprendre comment l'utiliser.
\\

Un tout petit problème rencontré était que mon programme renvoyait une erreur si le fichier qui contenait expression terminait sans retour à la ligne. J'ai simplement fait en sorte que "eof" renvoie un token EOL (fin de ligne) dans le lexer pour corriger ce problème.

\section {Compilateur}

La gestion des entiers n'a posé aucun souci et s'est faite rapidement.

Le premier problème était de réussir à empiler/dépiler des flottants, car il n'y avait pas de fonciton popsd ou pushsd, et que la plupart des opératoins sur la pile renvoyaient des erreurs de segmentation (donc impossible de savoir d'où vient l'erreur précisément) si on ne savait pas trop ce qu'on faisait.

Heureusement quelqu'un a trouvé la solution à ce problème et l'a partagée avec toute la classe, il fallait pour empiler déplacer les flottants 8 octets (car j'ai choisi le format double) avant la pile puis bouger le pointeur du haut de la pile 8 octets en arrière, et inversement pour dépiler.
\\
\\
Un autre petit problème lié aux flottants était l'abscence des opérations et registres pour ceux-ci dans la librairie x86-64, mais la façon dont la librairie été faite été assez claire et leur ajout n'était pas trop difficile (j'ai mis mes ajouts à la fin du fichier x86-64.ml).
\\
\\
Le dernier problème que j'ai eu concernait la fonction idivq pour les opérations de division entière et de modulo. Ces opérations semblaient mal fonctionner sur des entiers négatifs, ce qui s'expliquait par le fait que le bit de signe se retrouvait au milieu lors de la concatenation rdx::rax. Une rapide recherche sur internet m'a fait découvrir la commande cqto qui reglait ce problème (et qui était déjà dans la librairie x86-64). 


\newpage

\section{Autres problèmes}

Pour le bonus 1, j'ai voulu implémenter les fonctions factorielle et puissance avec des boucles, mais j'utilisais le saut conditionnel jns pour arrêter les boucles ce qui ne semblait pas foncitonner. J'ai toujours pas compris pourquoi mais j'ai utilisé je à la place, ce qui fonctionne très bien mais va faire boucler le programme indéfiniment si on veut faire la factorielle d'un nombre négatif.
\\
\\
J'ai aussi du mal avec l'utilisaiton de git sur le terminal, il y a toujours des problèmes de branches que je ne comprends pas trop, c'est peut-être lié au fait que j'ai cloné le dépot sur deux machines différentes à la fois. Du coup j'utilise l'interface web, qui me pose beaucoup moins de problèmes.

\end{document}
